\chapter{Ecuaciones Diferenciales parciales de la F'isica}

En F'isica es com'un encontrar sistemas descritos por \textit{campos}\footnote{Es decir, funciones que dependen de la posici'on y/o del tiempo.} ($\Psi$), que satisfacen ecuaciones diferenciales parciales (EDP's). Entre las m'as
frecuentes destacan las siguientes:
\begin{itemize}
\item Ecuaci'on de Laplace: 
\begin{equation}
\nabla^2\Psi=0, \qquad \Psi(\vec{x}).
\end{equation}
 Esta ecuaci'on aparece, por ejemplo, en el estudio de:
	\begin{itemize}
	\item Electrost'atica. El \textbf{potencial el'ectrico} $\phi$ en una \textit{regi'on sin cargas} satisface la ec. de Laplace.
	\item Hidrodin'amica. Un \textit{fluido irrotacional incompresible} en un \textit{movimiento estacionario} con campo de velocidad $\vec{v}=-\vec\nabla\Psi$ satisface
	\begin{equation}
	\frac{\partial\rho}{\partial t}+\vec\nabla (\rho\vec{v})=0 \quad\Rightarrow\quad
	\vec\nabla\cdot\vec{v}=-\nabla^2\Psi=0.
	\end{equation}
	\item Distribuci'on de Temperatura estacionaria: Aqu'i $\Psi=T(\vec{x},t)$ es el campo de temperaturas de un material, la ecuaci'on del Calor, ver \eqref{eccalor},  se reduce a la ec. de Laplace para $T(\vec{x})$.
	\item Gravitaci'on. An'alogo al caso electrost'atico, con $\Psi=\phi=\text{potencial gravitacional}$.
	\end{itemize}
\item Ecuaci'on de Poisson
\begin{equation}
\nabla^2\Psi=g(\vec{x}),
\end{equation}
donde $g(\vec{x})$ es una funci'on ``fuente'' conocida. Esta EDP es \textit{inhomog'enea}, y por lo tanto sus soluciones generales pueden escribirse como $\Psi=\Psi_{\rm h}+\Psi_{\rm p}$, donde $\Psi_{\rm h}$ es soluci'on de la ecuaci'on homogenea correspondiente (en este caso, la ec. de Laplace), y $\Psi_{\rm p}$ una \textbf{soluci'on particular} de la ec. de Poisson.

Por ejemplo, el potencial electrost'atico $\phi(\vec{x})$ satisface
\begin{equation}
\nabla^2\phi=-\frac{1}{\varepsilon_0}\rho(\vec{x}),
\end{equation}
donde $\varepsilon_0$ es la \textbf{permeabilidad del vac'io} y $\rho(\vec{x})$ la \textbf{densidad (volum'etrica) de carga el'ectrica}.
\item Ecuaci'on de Helmholtz: 
\begin{equation}
\nabla^2\Psi\pm k^2\Psi=0.
\end{equation}
Esta ecuaci'on, conocida tambi'en como la ecuaci'on de difusi'on independiente de tiempo, aparece en el estudio de
	\begin{itemize}
	\item Ondas el'asticas en s'olidos.
	\item Ac'ustica.
	\item Ondas electromagn'eticas.
	\item Reactores nucleares.
	\end{itemize}
\item Ecuaci'on de difusi'on dependiente del tiempo ($\alpha= $ difusividad t'ermica)
\begin{equation}\label{eccalor}
\nabla^2\Psi-\frac{1}{\alpha}\frac{\partial \Psi}{\partial t}=0 .
\end{equation}
\item Ecuaci'on de la onda dependiente del tiempo
\begin{equation}
\nabla^2\Psi-\frac{1}{v^2}\frac{\partial^2\Psi}{\partial t^2} =0,
\end{equation}
o bien 
\begin{equation}
\square\Psi=0, \qquad \square:=\frac{1}{v^2}\frac{\partial^2\ }{\partial t^2}-\nabla^2.
\end{equation}
Esta EDP aparece en modelos de:
\begin{itemize}
\item Ondas el'asticas en s'olidos, membranas, cuerdas, etc.
\item Ondas electromagn'eticas en regiones sin fuentes.
\item Ondas sonoras.
\end{itemize}

\item Ecuaci'on de Klein-Gordon:
\begin{equation}
\square\Psi-\frac{m^2c^2}{\hbar^2}\Psi=0.
\end{equation}

\item Ecuaci'on de Schr\"odinger:
\begin{equation}
-\frac{\hbar^2}{2m}\nabla^2\Psi+V(\vec{x})\Psi=i\hbar\frac{\partial\Psi}{\partial t}.
\end{equation}
\end{itemize}

Observaciones:

\begin{itemize}
\item Todas estas ecuaciones son \textit{lineales} en la funci'on desconocida.
\item Las ecuaciones fundamentales de la f'isica atmosf'erica son no-lineales, as'i como tambi'en las ecuaciones involucradas en los problemas de turbulencia.
\item Estas ecuaciones son casi todas de segundo orden excepto las ecuaciones de Maxwell y de Dirac que son de primer orden.
\end{itemize}

Las t'ecnicas generales para (intentar) resolver \textit{EDP lineales} que consideraremos en este curso son:
\begin{enumerate}
\item \textbf{M'etodo de separaci'on de variables}: La ecuaci'on diferencial parcial es desdoblada en ecuaciones diferenciales ordinarias lineales. La soluci'on de la EDP es construida como superposici'on de soluciones que son el producto de funciones dependientes de cada variable. Usando este m'etodo es posible reducir la ec. de la onda, la ec. del calor, y la ec. de Klein-Gordon a la ec. de Helmholtz.
\item \textbf{M'etodo de las transformadas integrales} (Fourier, Laplace, etc.), para resolver EDP inhomog'eneas.
\item \textbf{M'etodo de las funciones de Green}, para resolver EDP inhomog'eneas.

\end{enumerate}

\section{Clasificaci'on de E.D.P. lineales de segundo orden y Condiciones de Borde}

Una E.D.P. \textit{lineal de segundo orden} de la forma \cite{Hassani}
\begin{equation}\label{EDP2O}
\sum_{j=1}^m a_j(\vec{x})\frac{\partial^2\Psi}{\partial x_j^2}+F\left(\vec{x},\Psi,\vec\nabla\Psi\right)=0,
\end{equation}
o que pueda reducirse a (\ref{EDP2O}) por medio de alg'un cambio de variables, puede clasificarse en los siguientes tres tipos:
\begin{itemize}
\item \textbf{El'ipticas en $\vec{x}_0$:} si en el punto $\vec{x}_0$ todos los coeficientes $a_j(\vec{x}_0)$ son no-nulos y tienen \textit{el mismo signo}. Ejemplo cl'asico: Ec. de Laplace.
\item \textbf{Ultrahiperb'olicas en $\vec{x}_0$:} si en el punto $\vec{x}_0$ todos los coeficientes $a_j(\vec{x}_0)$ son no-nulos, pero \textit{no tienen el mismo signo}. Si s'olo uno de los coeficientes tiene signo diferente del resto, la E.D.P. es \textbf{hiperb'olica}. Ejemplo cl'asico: Ec. de onda.
\item \textbf{Parab'olica en $\vec{x}_0$:} si en el punto $\vec{x}_0$ al menos uno de los coeficientes $a_j(\vec{x}_0)$ se anula. Ejemplo cl'asico: Ec. de difusi'on del calor.
\end{itemize}
Si una E.D.P. de segundo orden es de un tipo dado en todos los puntos de su dominio, se dice simplemente que es de ese tipo (es decir, el tipo no cambia de punto a punto). Esto ocurre, en particular con las E.D.P.'s de segundo orden con coeficientes constantes.

\subsection{Condiciones de Borde y condiciones suficientes para determinar soluciones}
La soluci'on de une EDP requiere especificar informaci'on adicional a la ecuaci'on. Esta informaci'on recibe el nombre de \textbf{Condiciones de Borde} (C. de B.), o \textbf{condiciones de Frontera}, o bien \textbf{condiciones iniciales} y generalizan las conficiones iniciales necesarias para resulver una EDO.

En el caso de EDP's de segundo orden, existen tres tipos principales de C. de B.:

\begin{itemize}
\item C. de B. \textbf{tipo Dirichlet}: donde el valor de la funci'on ($\Psi$) es especificada en la frontera de la regi'on considerada.
\item C. de B. \textbf{tipo Neumann}: donde el valor de la \textbf{derivada normal de funci'on en la frontera} ($\partial\Psi/\partial t:=\hat{n}\cdot\vec\nabla\Psi$) es especificada.
\item C. de B. \textbf{tipo Cauchy}: donde se especifican \textit{simultaneamente} C. de B. tipo Dirichlet y Neumann en la frontera.
\end{itemize}
Por ejemplo, en electroest'atica (donde el potencial electrost'atico $\phi$ satisface la ecuaci'on de Poisson), una C. de B. tipo Dirichlet significa especificar un valor del potencial en la frontera, mientras que una C. de B. tipo Neumann equivale a especificar el valor de la componente del campo el'ectrico normal a la frontera. 

Existen teoremas de existencia y unicidad de soluciones de EDP's de segundo orden, dependiendo de si la ecuaci'on es parab'olica, hiperb'olica o el'iptica:

\textbf{Teorema:} Existe soluci'on 'unica en cada uno de los siguientes casos:

\begin{itemize}
\item E.D.P. el'ipticas +  C.de B. tipo \textit{Dirichlet o Neumann} sobre una (hiper)superficie \textit{cerrada}.
\item E.D.P. hiperb'olica + C.de B. tipo \textit{Cauchy} sobre una (hiper)superficie \textit{abierta}.
\item E.D.P. parab'olica + C.de B. tipo \textit{Dirichlet o Neumann} sobre una (hiper)superficie \textit{abierta}.
\end{itemize}

\subsection{Ejemplo: Difusi'on del calor $1-$dimensional}

Resolver la ecuaci'on de difusi´on del calor $1-$dimensional, empleando la transformada de Fourier espacial y temporal e imponiendo la siguiente condiciones de contorno
\begin{align}
\lim_{x \rightarrow \pm \infty}\psi(x,t)=0,\label{eq:bc1}\\
\psi(x,0)=\phi(x).\label{eq:bc2}
\end{align}

Como el problema es $1-$dimensional, la ecuaci'on a resolver es
\begin{align}
\alpha \frac{\partial^2 \psi}{\partial^2 x}-\frac{\partial \psi}{\partial t}=0,
\end{align}
as'i, aplicando la tranformada de Fourier espacial a la ecuaci'on precendente, se tiene que
\begin{align}
\alpha {\cal F}_{x}[\frac{\partial^2 \psi}{\partial^2 x}]-{\cal F}_{x}[\frac{\partial \psi}{\partial t}]=0,
\end{align}
adem'as por la condici'on de contorno \eqref{eq:bc1} se tiene que
\begin{align}
\frac{\partial^2 \psi}{\partial^2 x}=(i k)^2 \tilde{\psi}(k,t),
\end{align}
y como
\begin{align}
{\cal F}_{x}[\frac{\partial \psi}{\partial t}]=\frac{\partial}{\partial t}{\cal F}_{x}[\psi],
\end{align}
entonces
\begin{align}
-\alpha k^2 \tilde{\psi}(k,t)-\frac{\partial}{\partial t}\tilde{\psi}(k,t)=0,
\end{align}
por lo tanto
\begin{align}
\frac{\partial}{\partial t}\tilde{\psi}(k,t)=-\alpha \tilde{\psi}(k,t),
\end{align}
e integrando parcialmente en la variable $t$ se halla que
\begin{align}
\tilde{\psi}(k,t)=\tilde{\psi}(k,0)e^{-\alpha k^2 t}.
\end{align}
Por otro lado, empleando la condición \eqref{eq:bc2} vemos que
\begin{align}
{\cal F}_{x}[\psi(x,0)]={\cal F}_{x}[\psi(x)]\notag \\
\tilde{\psi}(k,0)=\tilde{\phi}(k),
\end{align}
en consecuencia
\begin{align}
\tilde{\psi}(k,t)=\tilde{\phi}(k)e^{-\alpha k^2 t}.
\end{align}
En efecto, la soluci'on del problema es
\begin{align}
\psi(x,t)=&\frac{1}{\sqrt{2\pi}}\int_{-\infty}^{\infty}\tilde{\psi}(k,t)e^{ikx} dk\notag \\
=&\frac{1}{\sqrt{2\pi}}\int_{-\infty}^{\infty}\tilde{\phi}(k)e^{-\alpha k^2 t}e^{ikx} dk,
\end{align}
con
\begin{align}
\tilde{\phi}(k)=\frac{1}{\sqrt{2\pi}}\int_{-\infty}^{\infty}\phi(x)e^{ikx}dx.
\end{align}
En particular, si suponemos
\begin{align}
\phi(x)=T_{0}\delta(x),
\end{align}
entonces
\begin{align}
\tilde{\phi}(k)=T_{0}{\cal F}_{x}[\delta(x)]=\frac{T_{0}}{\sqrt{2\pi}}.
\end{align}
Por consiguiente
\begin{align}
\psi(x,t)=\frac{T_{0}}{2\pi}\int_{-\infty}^{\infty}e^{-\alpha k^2 t} e^{ikx}dk
=\frac{T_{0}}{2\pi}\int_{-\infty}^{\infty}e^{-\alpha k^2 t+ikx}dk,
\end{align}
pero completando cuadrados vemos que
\begin{align}
-\alpha k^2 t+ikx=&-\alpha t \left( k^2-\frac{ikx}{\alpha t}\right)\notag \\
=&-\alpha t \left( k^2-\frac{ikx}{\alpha t}+\left(\frac{ix}{2\alpha t}\right)^2-\left(\frac{ix}{2\alpha t}\right)^2\right)\notag \\
=&-\alpha t\left( \left(k-\frac{ix}{2\alpha t} \right)^2 +\frac{x^2}{4 \alpha^2 t^2}\right)\notag \\
=& -\alpha t \left(k-\frac{ix}{2\alpha t} \right)^2-\frac{x^2}{4 \alpha t},
\end{align}
luego
\begin{align}
\psi(x,t)=&\frac{T_{0}}{2\pi}\int _{-\infty}^{\infty}e^{-\frac{x^2}{4 \alpha t}} e^{-\alpha t \left(k-\frac{ix}{2\alpha t} \right)^2} dk\notag \\
=&\frac{T_{0} e^{-\frac{x^2}{4 \alpha t}}}{2\pi}\int _{-\infty}^{\infty}e^{-\alpha t \left(k-\frac{ix}{2\alpha t} \right)^2} dk,
\end{align}
y haciendo el cambio de variables $y:=\sqrt{\alpha t}(k-ix/2\alpha t)$ se tiene que $dy=\sqrt{\alpha t} dk$, entonces
\begin{align}
\psi(x,t)=&\frac{T_{0} e^{-\frac{x^2}{4 \alpha t}}}{2\pi}\int _{-\infty}^{\infty}e^{-y^2} \frac{dy}{\sqrt{\alpha t}},
\end{align}
Por lo tanto, se ha hallado que\footnote{En \url{http://nbviewer.ipython.org/urls/dl.dropbox.com/s/0dgalwtzh6cr33m/difusion-calor.ipynb?dl=0} puede visualizar un notebook con la animaci'on de la soluci'on.}
\begin{align}
\psi(x,t)=&\frac{T_{0}}{2\sqrt{\pi \alpha t}}e^{-\frac{x^2}{4 \alpha t}}\qquad \forall t>0.
\end{align}
